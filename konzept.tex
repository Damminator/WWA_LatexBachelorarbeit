%!TEX root = thesis.tex

\chapter{Konzept}
\label{chapter-konzept}

Mit der Biologie als Vorbild wird nun durch eine passende Modellbildung eine für die Informationstechnik verwendbare Lösung gefunden. Durch eine grobe Verallgemeinerung wird das System vereinfacht – unter Erhaltung der wesentlichen Eigenschaften. Die Synapsen der Nervenzelle werden hierbei durch die Addition gewichteter Eingaben abgebildet, die Aktivierung des Zellkerns durch eine Aktivierungsfunktion mit Schwellenwert. Die Verwendung eines Addierers und Schwellenwerts findet sich so schon in der McCulloch-Pitts-Zelle von 1943. 

\section{Bestandteile}

Ein künstliches Neuron $j$ kann durch vier Basiselemente beschrieben werden:
\begin{enumerate}
  \item Wichtung: Die Gewichte $w_{ij}$ bestimmen den Grad des Einflusses, den die Eingaben des Neurons in der Berechnung der späteren Aktivierung einnehmen. Abhängig von den Vorzeichen der Gewichte kann eine Eingabe hemmend (inhibitorisch) oder erregend (exzitatorisch) wirken. Ein Gewicht von 0 markiert eine nicht existente Verbindung zwischen zwei Knoten.
  \item Übertragungsfunktion: Die Übertragungsfunktion $\sigma$ berechnet anhand der Wichtung der Eingaben die Netzeingabe des Neurons.
  \item Aktivierungsfunktion: Die Ausgabe des Neurons wird schließlich durch die Aktivierungsfunktion $\varphi$ bestimmt. Die Aktivierung wird beeinflusst durch die Netzeingabe aus der Übertragungsfunktion sowie einem Schwellenwert. 
  \item Schwellenwert: Das Addieren eines Schwellenwerts $\theta_j$ zur Netzeingabe verschiebt die gewichteten Eingaben. Die Bezeichnung bestimmt sich aus der Verwendung einer Schwellenwertfunktion als Aktivierungsfunktion, bei der das Neuron aktiviert wird, wenn der Schwellenwert überschritten ist. Die biologische Motivierung dabei ist das Schwellenpotential bei Nervenzellen. Mathematisch gesehen wird die Trennebene, die den Merkmalsraum auftrennt, durch einen Schwellenwert mit einer Translation verschoben. \cite{dendr}
\end{enumerate}

Durch einen Verbindungsgraphen werden folgende Elemente festgelegt:
\begin{enumerate}
\item Eingaben: Eingaben $x_{i}$ können einerseits aus dem beobachteten Prozess resultieren, dessen Werte dem Neuron übergeben werden, oder wiederum aus den Ausgaben anderer Neuronen stammen. Sie werden auch so dargestellt: Siehe \vref {fig-neuron-input}.
\item Aktivierung oder Ausgabe: Das Ergebnis der Aktivierungsfunktion wird analog zur Nervenzelle als Aktivierung $o_{j}$ des künstlichen Neurons $j$ bezeichnet.
\end{enumerate}

\begin{figure}
  \centering
  \pgfimage[width=.03\textwidth]{neuron_acting_as_input}
  \caption[Eingabe Neuron]{Darstellung der Eingaben eines Neurons}
  \label{fig-neuron-input}
\end{figure}

\section{Mathematische Definition}
Das künstliche Neuron als Modell wird in der Literatur meist auf dem folgenden Weg eingeführt:
\begin{Definition}[Definition]
  Zuerst wird die Netzeingabe $net_{j}$ des künstlichen Neurons $j$ durch

    \centerline {$net_j = \sum_{i = 1}^n x_iw_{ij}$}

definiert und damit die Aktivierung $o_j$ durch

	\centerline {$o_j = \varphi (net_j - \theta_j)$.}

Dabei ist $n$ Anzahl der Eingaben und $x_i$ die Eingabe $i$, die sowohl diskret als auch stetig sein kann. 
\end{Definition}

\section{on-Neuron}
Alternativ kann der Schwellenwert auch durch Hinzufügen eines weiteren Eingangs $x_0$, einem sogenannten on-Neuron oder auch Bias, dargestellt werden. Dieser hat den konstanten Wert $x_0 = 1$. Der Schwellenwert ist dann die Gewichtung dieses Eingangs $w_{0j} = - \theta_j$. Eine spezielle Behandlung des Schwellenwerts kann so entfallen und vereinfacht die Behandlung in den Lernregeln. Zusätzlich zu den echten Eingängen wird nun der des on-Neurons mit einberechnet:

	\centerline {$net_j = \sum_{i=0}^nx_iw_{ij}$}

Bei der Aktivierung $o_j$ kann damit auf eine spezielle Behandlung des Schwellenwerts verzichtet werden:

	\centerline {$o_j = \varphi (net_j)$}

Diese Unterschiede sind in \ref{fig-perceptron} zu sehen.
 
\begin{figure}
	\centering
	\pgfimage[width=0.5\textwidth]{Perceptron-bias}
	\caption[Perceptron]{Zwei funktional identische Neuronen, das linke ohne und das rechte mit „on“-Neuron oder Bias}
	\label{fig-perceptron}

\end{figure}



