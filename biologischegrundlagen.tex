%!TEX root = thesis.tex

\chapter{Biologische Grundlagen}
\label{chapter-basics}

Motiviert sind künstliche Neuronen durch die Nervenzellen der Säugetiere, die auf die Aufnahme und Verarbeitung von Signalen spezialisiert sind. Über Synapsen werden Signale elektrisch oder chemisch an andere Nervenzellen oder Effektorzellen (etwa zur Muskelkontraktion) weitergeleitet. Eine Nervenzelle besteht aus dem Zellkörper, Axon und den Dendriten. Dendriten sind kurze Zellfortsätze, die stark verzweigt für die Aufnahme von Signalen anderer Nervenzellen oder Sinneszellen sorgen. Das Axon funktioniert als Signalausgang der Zelle und kann eine Länge bis 1 m erreichen. Der Übergang der Signale erfolgt an den Synapsen, welche erregend oder hemmend wirken können. Die Dendriten der Nervenzelle leiten die eingehenden elektrischen Erregungen an den Zellkörper weiter. Erreicht die Erregung einen gewissen Grenzwert und übersteigt ihn, entlädt sich die Spannung und pflanzt sich über das Axon fort (Alles-oder-nichts-Gesetz). Die Verschaltung dieser Nervenzellen bildet die Grundlage für die geistige Leistung des Gehirns. Das Zentralnervensystem des Menschen besteht nach Schätzungen aus $10^{10}$ bis $10^{12}$ Nervenzellen, die durchschnittlich 10.000 Verbindungen besitzen – das menschliche Gehirn kann also mehr als $10^{14}$ Verbindungen besitzen.\cite{neuroa} Das Aktionspotential im Axon kann sich mit einer Geschwindigkeit bis zu 100 m/s fortpflanzen. 
