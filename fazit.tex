%!TEX root = thesis.tex

\chapter{Zusammenfassung und Ausblick}
\label{chapter-fazit}

Im Vergleich zu Logikgattern zeigt sich auch die Effizienz von Neuronen. Während Gatter im Nanosekunden-Bereich (${10}^{- 9}$) schalten, unter einem Energieverbrauch von ${10}^{-6}$ Joule (Daten von 1991), reagieren Nervenzellen im Millisekunden-Bereich (${10}^{-3}$) und verbrauchen lediglich eine Energie von ${10}^{-16}$ Joule. Trotz der augenscheinlich geringeren Werte in der Verarbeitung durch Nervenzellen können rechnergestützte Systeme nicht an die Fähigkeiten biologischer Systeme heranreichen. Die Leistung neuronaler Netze wird ebenfalls durch die 100-Schritt-Regel demonstriert: Die visuelle Erkennung beim Menschen findet in maximal 100 sequentiellen Verarbeitungsschritten statt – die meist sequentiell arbeitenden Rechner erbringen keine vergleichbare Leistung. Die Vorteile und Eigenschaften von Nervenzellen motivieren das Modell der künstlichen Neuronen. Viele Modelle und Algorithmen zu künstlichen neuronalen Netzen entbehren dennoch einer direkt plausiblen, biologischen Motivierung. Dort findet sich diese nur im Grundgedanken der abstrakten Modellierung der Nervenzelle.