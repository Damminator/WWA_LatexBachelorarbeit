%!TEX root = thesis.tex

\cleardoublepage
\thispagestyle{plain}

\pdfbookmark{Kurzfassung}{kurzfassung}
\paragraph{Kurzfassung}Ein künstliches Neuron bildet die Basis für das Modell der künstlichen neuronalen Netze, ein Modell aus der Neuroinformatik, das durch biologische neuronale Netze motiviert ist. Als konnektionistisches Modell bilden sie in einem Netzwerk aus künstlichen Neuronen ein künstliches neuronales Netz und können so beliebig komplexe Funktionen approximieren, Aufgaben erlernen und Probleme lösen, bei denen eine explizite Modellierung schwierig bis nicht durchzuführen ist. Beispiele sind die Gesichts- und Spracherkennung. Als Modell aus dem biologischen Vorbild der Nervenzelle entstanden, kann es mehrere Eingaben verarbeiten und entsprechend über seine Aktivierung reagieren. Dazu werden die Eingaben gewichtet an eine Ausgabefunktion übergeben, welche die Neuronenaktivierung berechnet. Ihr Verhalten wird ihnen im Allgemeinen durch Einlernen unter Verwendung eines Lernverfahrens gegeben.

\cleardoublepage
\thispagestyle{plain}

\foreignlanguage{english}{%
\pdfbookmark{Abstract}{abstract}
\paragraph{Abstract}An artificial neuron forms the basis for the model of artificial neural networks, a model of neuroinformatics motivated by biological neural networks. As a connectivist model, they form an artificial neural network in a network of artificial neurons and can thus approximate arbitrarily complex functions, learn tasks, and solve problems where an explicit modeling is difficult to perform. Examples are facial and speech recognition. As a model developed from the biological model of the nerve cell, it can process several inputs and react accordingly via its activation. For this purpose, the inputs are passed weighted to an output function, which calculates the neuron activation. Their behavior is generally given to them by learning-in using a learning method.
}