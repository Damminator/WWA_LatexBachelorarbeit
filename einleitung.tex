%!TEX root = thesis.tex

\chapter{Einleitung}
Die Anfänge der künstlichen Neuronen gehen auf Warren McCulloch und Walter Pitts im Jahr 1943 zurück. Sie zeigen an einem vereinfachten Modell eines Neuronalen Netzes, der McCulloch-Pitts-Zelle, dass diese logische und arithmetische Funktionen berechnen kann.

Die Hebbsche Lernregel wird im Jahr 1949 von Donald Hebb beschrieben. Aufbauend auf der medizinischen Forschung von Santiago Ramón y Cajal, der bereits 1911 die Existenz von Synapsen nachgewiesen hat, werden nach dieser Regel wiederholt aktive Verbindungen zwischen Nervenzellen gestärkt. Die Verallgemeinerung dieser Regel wird auch in den heutigen Lernverfahren noch verwendet.

Eine wichtige Arbeit kommt im Jahre 1958 mit dem Konvergenztheorem über das Perzeptron heraus. Dort zeigt Frank Rosenblatt, dass es mit dem angegebenen Lernverfahren alle Lösungen einlernen kann, die mit diesem Modell repräsentierbar sind.

Jedoch zeigen die Kritiker Marvin Minsky und Seymour Papert 1969, dass ein einstufiges Perzeptron eine XOR-Verknüpfung nicht repräsentieren kann, weil die XOR-Funktion nicht linear separierbar (linear trennbar) ist, erst spätere Modelle können diesen Missstand beheben. Die so gezeigte Grenze in der Modellierung führt zunächst zu einem abnehmenden Interesse an der Erforschung der künstlichen neuronalen Netze sowie zu einer Streichung von Forschungsgeldern.

Ein Interesse an künstlichen Neuronalen Netzen kommt erst wieder auf, als John Hopfield die Hopfield-Netze 1985 bekannt macht und zeigt, dass sie in der Lage sind Optimierungsprobleme zu lösen, wie das Problem des Handlungsreisenden.\cite{neuro} Ebenfalls führt die Arbeit zum Backpropagation-Verfahren von David E. Rumelhart, Geoffrey E. Hinton und Ronald J. Williams ab 1986 zu einer Wiederbelebung der Erforschung dieser Netze.

Heute werden solche Netze in vielen Forschungsbereichen verwendet.
